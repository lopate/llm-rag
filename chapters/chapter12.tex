\chapter{Равномерная сходимость функциональных последовательностей и рядов. Непрерывность, интегрируемость и дифференцируемость суммы равномерно сходящегося ряда.}

\section{Равномерная сходимость функциональных последовательностей}

Пусть задана последовательность функций $\{f_n\}$
\begin{equation}
	\label{ch12:p1:eq:1}
	f_n(x),\ n\in \bbN,
\end{equation}
определенных на некотором множестве $E$.

\begin{defn}
	Функциональная последовательность $\{f_n(x)\}_{n=1}^\infty$ называется \textit{сходящейся в точке} $x_0 \in E$,
	если числовая последовательность $\{f_n(x_0)\}_{n=1}^\infty$ сходится.

	Последовательность $\{f_n\}$ называется сходящейся на множестве $E$, если она сходится в каждой точке множества $E$.

	Если
	$$
		\lim_{n \to \infty} f_n(x) = f(x)\quad \forall x\in E,
	$$
	т. е.
	\begin{equation}
		\label{ch12:p1:eq:2}
		\forall x \in E \quad \forall \epsilon > 0 \quad \exists N \in \bbN \cquad
			\forall n \ge N \quad |f_n(x) - f(x)| \le \epsilon
	\end{equation}
	то говорят, что \textit{последовательность \eqref{ch12:p1:eq:1} на множестве $E$ сходится к функции} $f(x)$,
	и пишут <<$f_n(x) \to f(x)$ на $E$>> или <<$f_n(x) \xrightarrow{E} f(x)$ при $n \to +\infty$>>.

	Эта функция $f(x)$ называется пределом или предельной функцией последовательности.
	В этом случае иногда говорят, что \textit{последовательность $\{f_n\}$ сходится к функции~$f(x)$ поточечно}.
\end{defn}

\begin{defn}
	Будем говорить, что функциональная последовательность $\{f_n(x)\}_{n=1}^\infty$ \textit{равномерно сходится} к функции $f(x)$
	на множестве $E$ и писать <<$f_n(x) \rightrightarrows f(x)$ на $E$>>, если
	\begin{equation}
		\label{ch12:p1:eq:3}
		\forall \epsilon > 0 \quad \exists N \in \bbN \cquad
			\forall n \ge N \quad \forall x \in E \quad |f_n(x) - f(x)| \le \epsilon
	\end{equation}
\end{defn}

Отличие условий \eqref{ch12:p1:eq:2} и \eqref{ch12:p1:eq:3} состоит в том,
что в условии~\eqref{ch12:p1:eq:2} число $N$ свое для каждого $x$, а в условии~\eqref{ch12:p1:eq:3}
число $N$ не зависит от $x$. Поэтому из равномерной сходимости следует поточечная сходимость.

\begin{thm}[Критерий равномерной сходимости]
	Функциональная последовательность $\{f_n(x)\}_{n=1}^\infty$ равномерно сходится к функции $f(x)$
	на множестве $E$ при $n \to +\infty$ тогда и только тогда, когда
	$$
		\sup_{x \in E} |f_n(x) - f(x)| \to 0 \quad \text{при} \quad n \to +\infty
	$$
\end{thm}
\begin{proof}
	Поскольку условие $\forall x \in E \quad |f_n(x) - f(x)| \le \epsilon$ эквивалентно
	условию $\sup_{x \in E}\limits |f_n(x) - f(x)| \le \epsilon$, то условие~\eqref{ch12:p1:eq:3}
	эквивалентно условию
	$$
		\forall \epsilon > 0 \quad \exists N \in \bbN \cquad \forall n \ge N \quad
			\sup_{x \in E} |f_n(x) - f(x)| \le \epsilon,
	$$
	т. е. $\sup_{x \in E}\limits |f_n(x) - f(x)| \to 0$ при  $n \to +\infty$
\end{proof}
\begin{cons}
	Последовательность $\{f_n(x)\}$ сходится к функции $f(x)$ равномерно на множестве $E$
	тогда и только тогда, когда существует числовая последовательность $\{a_n\}$:
	\begin{equation}
		\label{ch12:p1:eq:4}
		\forall x \in E \quad \forall n \in \bbN \quad |f_n(x) - f(x)| \le a_n \quad
			\text{и} \quad \lim_{n \to \infty} a_n = 0
	\end{equation}
\end{cons}

\begin{thm}[Критерий Коши]
	\label{ch12:th:cauchy_criteria}
	Последовательность $\{f_n(x)\}$ сходится к функции $f(x)$ равномерно на множестве $E$
	тогда и только тогда, когда выполняется условие Коши равномерной сходимости последовательности:
	\begin{equation}
		\label{ch12:p1:eq:cauchy_cond}
		\forall \epsilon > 0 \quad \exists N \in \bbN \cquad
			\forall n \ge N \quad \forall p \in \bbN \quad \forall x \in E \quad
			|f_n(x) - f_{n+p}(x)| \le \epsilon
	\end{equation}
\end{thm}
\begin{proof}\leavevmode
\begin{itemize}[wide, labelwidth=!, labelindent=0pt]
	\item[$\Longrightarrow$:]
	Пусть $f_n(x) \rightrightarrows f(x)$ при $n \to \infty$, тогда 
	$$
		\forall \epsilon > 0 \quad \exists N \in \bbN \cquad
			\forall n \ge N \quad \forall x \in E \quad |f_n(x) - f(x)| \le \frac{\epsilon}{2}
	$$
	Поскольку $\forall p \in \bbN \quad n + p > n \ge N$, то
	$\forall x \in E \quad |f_{n+p} - f(x)| \le \frac{\epsilon}{2}$.\\
	Следовательно,
	$$
		\forall x \in E \quad |f_n(x) - f_{n+p}(x)| \le |f_n(x) - f(x)| + |f_{n+p}(x) - f(x)| \le
			\frac{\epsilon}{2} + \frac{\epsilon}{2} = \epsilon,
	$$
	т. е. выполняется условие~\eqref{ch12:p1:eq:cauchy_cond}.

	\item[$\Longleftarrow$:]
	Пусть выполняется условие~\eqref{ch12:p1:eq:cauchy_cond}. Следовательно,
	$$
		\forall x \in E \quad \forall \epsilon > 0 \quad \exists N \in \bbN \cquad
			\forall n \ge N \quad \forall p \in \bbN \quad
			|f_n(x) - f_{n+p}(x)| \le \epsilon
	$$
	т. е. для любого фиксированного $x \in E$ выполняется условие Коши сходимости
	числовой последовательности $\{f_n(x)\}$. В силу критерия Коши для числовых
	последовательностей $\forall x \in E$ последовательность $\{f_n(x)\}$ сходится.
	Обозначим $f(x) = \lim_{n \to \infty}\limits f_n(x)$.

	Перепишем условие~\eqref{ch12:p1:eq:cauchy_cond} в виде
	$$
		\forall \epsilon > 0 \quad \exists N \in \bbN \cquad
			\forall n \ge N \quad \forall x \in E \quad \forall p \in \bbN \quad 
			|f_n(x) - f_{n+p}(x)| \le \epsilon
	$$
	и рассмотрим отдельно условие $\forall p \in \bbN \quad |f_n(x) - f_{n+p}(x)| \le \epsilon$.
	Поскольку $\lim_{p \to \infty}\limits |f_n(x) - f_{n+p}(x)| = |f_n(x) - f(x)|$, то
	по теореме о предельном переходе в неравенствах $|f_n(x) - f(x)| \le \epsilon$.

	Итак, из условия~\eqref{ch12:p1:eq:cauchy_cond} следует, что
	$$
		\forall \epsilon > 0 \quad \exists N \in \bbN \cquad
			\forall n \ge N \quad \forall x \in E \quad |f_n(x) - f(x)| \le \epsilon,
	$$
	т. е. $f_n(x) \rightrightarrows f(x)$ при $n \to \infty$.
\end{itemize}
\end{proof}


\section{Равномерная сходимость функциональных рядов}

\begin{defn}
	Пусть на множестве $E$ задана функциональная последовательность $\{u_k(x)\}_{k=1}^\infty$.
	Функциональный ряд
	\begin{equation}
		\label{ch12:p2:eq:fun_series}
		\sum_{k=1}^{\infty} u_k(x)
	\end{equation}
	называется \textit{равномерно сходящимся} на множестве $E$, если последовательность его
	частичных сумм $S_n(x) = \sum_{k=1}^{n}\limits u_k(x)$ сходится равномерно на множестве $E$
	к сумме $S(x)$ этого ряда. Аналогично определяется \textit{поточечная сходимость} ряда.
\end{defn}

Поскольку из равномерной сходимости последовательности следует поточечная сходимость последовательности,
то из равномерной сходимости ряда следует поточечная сходимость этого ряда.

\begin{defn}
	\textit{Остатком} поточечно сходящегося ряда $\sum_{k=1}^{\infty}\limits u_k(x)$ называется
	$$
		r_n(x) = S(x) - S_n(x) = \sum_{k=n+1}^{\infty} u_k(x)
	$$
\end{defn}

Непосредственно из определения равномерной сходимости ряда и критерия равномерной сходимости
функциональной последовательности следует

\begin{thm}[Критерий равномерной сходимости ряда]
	\label{ch12:th:ce_series_criteria}
	Поточечно сходящийся функциональный ряд $\sum_{k=1}^{\infty}\limits u_k(x)$ сходится
	равномерно на множестве $E$ тогда и только тогда, когда
	$$
		r_n(x) \rightrightarrows 0 \quad \text{на } E \quad \text{при } n \to \infty,
	$$
	$$
		\text{т. е.} \quad \sup_{x \in E} |r_n(x)| \to 0 \quad \text{при } n \to \infty.
	$$
\end{thm}

\begin{thm}[Критерий Коши]
	\label{ch12:th:cauchy_criteria_series}
	Ряд $\sum_{k=1}^{\infty}\limits u_k(x)$ сходится равномерно на множестве $E$ тогда
	и только тогда, когда выполняется условие Коши равномерной сходимости ряда
	\begin{equation}
		\label{ch12:p1:eq:cauchy_cond_series}
		\forall \epsilon > 0 \quad \exists N \in \bbN \cquad
			\forall n \ge N \quad \forall p \in \bbN \quad \forall x \in E \quad
			|\sum_{k=n+1}^{n+p} u_k(x)| \le \epsilon
	\end{equation}
\end{thm}
\begin{proof}
	Необходимо применить критерий Коши равномерной сходимости последовательности
	(теорема~\ref{ch12:th:cauchy_criteria}) к последовательности частичных сумм ряда.
\end{proof}
\begin{cons}[Необходимое условие равномерной сходимости ряда]
	Если ряд $\sum_{k=1}^{\infty}\limits u_k(x)$ сходится равномерно на множестве $E$,
	то
	$$
		u_n(x) \rightrightarrows 0 \quad \text{на } E \quad \text{при } n \to \infty
	$$
\end{cons}


\subsection*{Признаки равномерной сходимости функциональных рядов}

\begin{thm}[Обобщенный признак сравнения]
	Пусть
	$$
		\forall k \in \bbN \quad \forall x \in E \quad |u_k(x)| \le v_k(x)
	$$
	и ряд $\sum_{k=1}^{\infty}\limits v_k(x)$ сходится равномерно на множестве $E$.
	Тогда ряд $\sum_{k=1}^{\infty}\limits u_k(x)$ сходится равномерно на множестве $E$.
\end{thm}
\begin{cons}
	Если ряд $\sum_{k=1}^{\infty}\limits |u_k(x)|$ сходится равномерно на множестве $E$,
	то $\sum_{k=1}^{\infty}\limits u_k(x)$ сходится равномерно на множестве $E$.
\end{cons}

\begin{thm}[Признак Вейерштрасса]
	Если
	$$
		\forall k \in \bbN \quad \forall x \in E \quad |u_k(x)| \le a_k
	$$
	и числовой ряд $\sum_{k=1}^{\infty}\limits a_k$ сходится, то
	ряд $\sum_{k=1}^{\infty}\limits u_k(x)$ сходится равномерно на множестве $E$.
\end{thm}

\begin{thm}[Признак Дирихле]
	Пусть на множестве $E$ заданы две функциональные последовательности
	$\{a_k(x)\}_{k=1}^\infty$ и $\{b_k(x)\}_{k=1}^\infty$, удовлетворяющие условиям:
	\begin{enumerate}
		\item последовательность частичных сумм $A_n(x) = \sum_{k=1}^{n}\limits a_k(x)$\\
			ряда $\sum_{k=1}^{\infty}\limits a_k(x)$ равномерно ограничена, т. е.
			существует число $C$, не зависящее от $x$ и от $n$:
			$$
				\forall n \in \bbN \quad \forall x \in E \quad |A_n(x)| \le C
			$$
		\item $b_k(x) \rightrightarrows 0$ на $E$ при $k \to \infty$
		\item $\forall x \in E \quad \forall k \in \bbN \quad b_{k+1}(x) \le b_k(x)$
	\end{enumerate}
	Тогда ряд $\sum_{k=1}^{\infty}\limits a_k(x)b_k(x)$ равномерно сходится на $E$.
\end{thm}

\begin{thm}[Признак Лейбница]
	Пусть $\forall k \in \bbN \quad \forall x \in E \quad b_{k+1}(x) \le b_k(x)$
	и $b_k(x) \rightrightarrows 0$ на $E$ при $k \to \infty$. Тогда ряд Лейбница
	$\sum_{k=1}^{\infty}\limits (-1)^k b_k(x)$ равномерно сходится.
\end{thm}

\begin{thm}[Признак Абеля]
	Пусть на множестве $E$ заданы две функциональные последовательности
	$\{a_k(x)\}_{k=1}^\infty$ и $\{b_k(x)\}_{k=1}^\infty$, удовлетворяющие условиям:
	\begin{enumerate}
		\item ряд $\sum_{k=1}^{\infty}\limits a_k(x)$ равномерно сходится на множестве $E$
		\item последовательность $\{b_k(x)\}$ равномерно ограничена, т. е.
			$$
				\exists C \in \bbR \cquad \forall k \in \bbN \quad \forall x \in E
					\quad |b_k(x)| \le C
			$$
		\item $\forall x \in E \quad \forall k \in \bbN \quad b_{k+1}(x) \le b_k(x)$
	\end{enumerate}
	Тогда ряд $\sum_{k=1}^{\infty}\limits a_k(x)b_k(x)$ равномерно сходится на $E$.
\end{thm}


\section{Непрерывность, интегрируемость и дифференцируемость суммы равномерно сходящегося ряда}

\begin{thm}[О непрерывности суммы ряда]
	\label{ch12:th:continuity}
	Если все члены ряда~\eqref{ch12:p2:eq:fun_series} --- непрерывные на отрезке $[a, b]$ функции,
	а ряд~\eqref{ch12:p2:eq:fun_series} сходится равномерно на $[a, b]$, то его сумма $S(x)$
	также непрерывна на отрезке $[a, b]$.
\end{thm}
\begin{proof}
	Пусть $x_0$ --- произвольная точка отрезка $[a, b]$. Для определенности будем считать,
	что $x_0 \in (a, b)$.

	Нужно доказать, что функция
	$$
		S(x) = \sum_{n=1}^{\infty} u_n(x)
	$$
	непрерывна в точке $x_0$, т. е.
	\begin{equation}
		\label{ch12:eq:1}
		\forall \epsilon > 0 \quad \exists \delta = \delta(\epsilon) > 0 \cquad
			\forall x \in U_{\delta}(x_0) \quad |S(x) - S(x_0)| < \epsilon,
	\end{equation}
	где $U_{\delta}(x_0) = (x_0 - \delta, x_0 + \delta) \subset [a, b]$.

	По условию $S_n(x) \rightrightarrows S(x)$, $x \in [a, b]$,
	где $S_n(x) = \sum_{k=1}^{n}\limits u_k(x)$, т. е.
	\begin{equation}
		\label{ch12:eq:2}
		\forall \epsilon > 0 \quad \exists N_{\epsilon} \cquad \forall n \ge N_{\epsilon} \quad
			\forall x \in [a, b] \quad |S(x) - S_n(x)| < \frac{\epsilon}{3}
	\end{equation}
	Фиксируем номер $n_0 \ge N_{\epsilon}$. Тогда из~\eqref{ch12:eq:2} при $n = n_0$ получаем
	\begin{equation}
		\label{ch12:eq:3}
		|S(x) - S_{n_0}(x)| < \frac{\epsilon}{3}
	\end{equation}
	и, в частности, при $x = x_0$ находим
	\begin{equation}
		\label{ch12:eq:4}
		|S(x_0) - S_{n_0}(x_0)| < \frac{\epsilon}{3}
	\end{equation}
	Функция $S_{n_0}(x)$ непрерывна в точке $x_0$ как сумма конечного числа непрерывных функций
	$u_k(x)$, $k = \overline{1, n_0}$. По определению непрерывности
	\begin{equation}
		\label{ch12:eq:5}
		\forall \epsilon > 0 \quad \exists \delta = \delta(\epsilon) > 0 \cquad
			\forall x \in U_\delta(x_0) \subset [a, b] \quad |S_{n_0}(x) - S_{n_0}(x_0)| < \frac{\epsilon}{3}
	\end{equation}
	Воспользуемся равенством
	$$
		S(x) - S(x_0) = (S(x) - S_{n_0}(x)) + (S_{n_0}(x) - S_{n_0}(x_0)) + (S_{n_0}(x_0) - S(x_0)).
	$$
	Из этого равенства, используя оценки~\eqref{ch12:eq:3}--\eqref{ch12:eq:5}, получаем
	$$
		|S(x) - S(x_0)| = |S(x) - S_{n_0}(x)| + |S_{n_0}(x) - S_{n_0}(x_0)| + |S_{n_0}(x_0) - S(x_0)| < \epsilon
	$$
	для любого $x \in U_\delta(x_0) \subset [a, b]$, т. е. справедливо утверждение~\eqref{ch12:eq:1}.

	Так как $x_0$ --- произвольная точка отрезка $[a, b]$, то функция $S(x)$ непрерывна на отрезке $[a, b]$.
\end{proof}

\begin{cons}
	Согласно теореме~\ref{ch12:th:continuity}
	$$
		\lim_{x \to x_0} \sum_{n = 1}^{\infty} u_n(x) = \sum_{n = 1}^{\infty} \lim_{x \to x_0} u_n(x),
	$$
	т. е. при условиях теоремы~\eqref{ch12:th:continuity} возможен почленный предельный переход.
\end{cons}


\begin{thm}[О почленном интегрировании ряда]
	\label{ch12:th:integration}
	Если все члены ряда~\eqref{ch12:p2:eq:fun_series} --- непрерывные на отрезке $[a, b]$ функции,
	а ряд~\eqref{ch12:p2:eq:fun_series} сходится равномерно на $[a, b]$, то ряд
	\begin{equation}
		\label{ch12:eq:6}
		\sum_{n=1}^{\infty} \int_a^x u_n(t)dt
	\end{equation}
	также равномерно сходится на $[a, b]$, и если
	\begin{equation}
		\label{ch12:eq:7}
		S(x) = \sum_{n=1}^{\infty} u_n(x),
	\end{equation}
	то
	\begin{equation}
		\label{ch12:eq:8}
		\int_a^x S(t)dt = \sum_{n=1}^{\infty} \int_a^x u_n(t)dt, \quad x \in [a, b],
	\end{equation}
	т. е. ряд~\eqref{ch12:eq:7} можно почленно интегрировать
\end{thm}
\begin{proof}
	По условию ряд~\eqref{ch12:eq:7} сходится равномерно к $S(x)$ на отрезке $[a, b]$,
	т. е. $S_n(x) = \sum_{k=1}^{n}\limits u_k(x) \rightrightarrows S(x), x \in [a, b]$.
	Это означает, что
	\begin{equation}
		\label{ch12:eq:9}
		\forall \epsilon > 0 \quad \exists N_{\epsilon} \cquad \forall n \ge N_{\epsilon} \quad
			\forall t \in [a, b] \quad |S(t) - S_n(t)| < \frac{\epsilon}{b - a}
	\end{equation}
	Пусть $\sigma(x) = \int_a^x\limits S(t)dt$, а $\sigma_n(x) = \sum_{k=1}^n\limits \int_a^x\limits u_k(t)dt$ ---
	$n$-я частичная сумма ряда~\eqref{ch12:eq:6}.

	Функции $u_k(x), k \in \bbN$, по условию непрерывны на отрезке $[a, b]$, и поэтому
	они интегрируемы на $[a, b]$. Функция $S(x)$ также интегрируема на $[a, b]$, так как
	она непрерывна на этом отрезке (теорема~\ref{ch12:th:continuity}).
	Используя свойства интеграла, получаем
	$$
		\sigma_n(x) = \int_a^x \sum_{k=1}^n u_k(t)dt = \int_a^x S_n(t)dt.
	$$
	Следовательно,
	$$
		\sigma(x) - \sigma_n(x) = \int_a^x (S(t) - S_n(t))dt,
	$$
	откуда в силу условия~\eqref{ch12:eq:9} получаем
	$$
		|\sigma(x) - \sigma_n(x)| < \frac{\epsilon}{b - a} \int_a^x dt = \frac{\epsilon}{b - a} (x - a) \le \epsilon,
	$$
	причем это неравенство выполняется для всех $n \ge N_{\epsilon}$ и для всех $x \in [a, b]$.
	Это означает, что ряд~\eqref{ch12:eq:6} сходится равномерно на отрезке $[a, b]$,
	и выполняется равенство~\eqref{ch12:eq:8}.
\end{proof}


\begin{thm}[О почленном дифференцировании ряда]
	Если функции $u_n(x), n \in \bbN$, имеют непрерывные производные на отрезке $[a, b]$, ряд
	\begin{equation}
		\label{ch12:eq:10}
		\sum_{n=1}^{\infty} u'_n(x)
	\end{equation}
	сходится равномерно на отрезке $[a, b]$, а ряд
	\begin{equation}
		\label{ch12:eq:11}
		\sum_{n=1}^{\infty} u_n(x)
	\end{equation}
	сходится хотя бы в одной точке $x_0 \in [a, b]$, т. е. сходится ряд
	\begin{equation}
		\label{ch12:eq:12}
		\sum_{n=1}^{\infty} u_n(x_0),
	\end{equation}
	то ряд~\eqref{ch12:eq:11} сходится равномерно на отрезке $[a, b]$,
	и его можно почленно дифференцировать, т. е.
	\begin{equation}
		\label{ch12:eq:13}
		S'(x) = \sum_{n=1}^{\infty} u'_n(x),
	\end{equation}
	где
	\begin{equation}
		\label{ch12:eq:14}
		S(x) = \sum_{n=1}^{\infty} u_n(x)
	\end{equation}
\end{thm}
\begin{proof}
	Обозначим через $\tau(x)$ сумму ряда~\eqref{ch12:eq:10}, т. е.
	\begin{equation}
		\label{ch12:eq:15}
		\tau(x) = \sum_{n=1}^{\infty} u'_n(x)
	\end{equation}
	По теореме~\ref{ch12:th:integration} ряд~\eqref{ch12:eq:15} можно почленно интегрировать, т. е.
	\begin{equation}
		\label{ch12:eq:16}
		\int_{x_0}^x \tau(t)dt = \sum_{n=1}^{\infty} \int_{x_0}^x u'_n(t)dt,
	\end{equation}
	где $x_0, x \in [a, b]$, причем ряд~\eqref{ch12:eq:16} сходится равномерно на отрезке $[a, b]$.
	Так как $\int_{x_0}^x\limits u'_n(t)dt = u_n(x) - u_n(x_0)$, то равенство~\eqref{ch12:eq:16}
	можно записать в виде
	\begin{equation}
		\label{ch12:eq:17}
		\int_{x_0}^x \tau(t)dt = \sum_{n=1}^{\infty} \upsilon_n(x),
	\end{equation}
	где
	\begin{equation}
		\label{ch12:eq:18}
		\upsilon_n(x) = u_n(x) - u_n(x_0).
	\end{equation}
	Ряд~\eqref{ch12:eq:17} сходится равномерно, а ряд~\eqref{ch12:eq:12} сходится
	(а значит, и равномерно сходится на $[a, b]$). Поэтому ряд~\eqref{ch12:eq:11}
	сходится равномерно на $[a, b]$ как разность равномерно сходящихся рядов.

	Из равенств \eqref{ch12:eq:17}, \eqref{ch12:eq:18} и \eqref{ch12:eq:14} следует, что
	\begin{equation}
		\label{ch12:eq:19}
		\int_{x_0}^x \tau(t)dt = S(x) - S(x_0)
	\end{equation}
	Так как функция $\tau(t)$ непрерывна на отрезке $[a, b]$ по теореме~\ref{ch12:th:continuity},
	то в силу свойств интеграла с переменным верхним пределом левая часть равенства~\eqref{ch12:eq:19}
	имеет производную, которая равна $\tau(x)$. Следовательно, правая часть~\eqref{ch12:eq:19} ---
	дифференцируемая функция, а её производная равна $S'(x)$. Итак, доказано, что $\tau(x) = S'(x)$,
	т. е. справедливо равенство~\eqref{ch12:eq:13} для всех $x \in [a, b]$
\end{proof}