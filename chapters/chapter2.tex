\chapter{Ограниченность функции, непрерывной на отрезке, достижение точных верхней и нижней граней.}\label{chapter2}

\section{Определение функции}
\begin{defn}
Пусть $D$ и $Y$ "--- два произвольных множества, и задано некоторое правило $f$, которое каждому элементу $x\in D$ ставит в соответствие один и только один некоторый элемент $y=f(x)$ из $Y$. Тогда множество всевозможных пар $(x, f(x))$, $x\in D$, называется \textit{функцией}\rindex{функция} и обозначается либо просто $f$, либо $f(x),$ $x\in D$, либо $y=f(x),\ x\in D$, либо, например, $f\colon D \to Y$.
\end{defn}

Функция называется \textit{числовой}, если ее значениями являются действительные числа: $Y \subset \bbR$.
\begin{itemize}[wide, labelwidth=!, labelindent=0pt]
\item
Элемент $x \in D$ называется \textit{аргументом}\rindex{аргумент}, или \textit{независимым перемененным}, элемент $y=f(x) \in Y$ "--- \textit{значением функции}, или \textit{зависимым переменным}. Таким образом, $f(x)$ может обозначать как значении функции $f$ на элементе $x$, так и саму функцию $f$.
\item
Множество $D$ называется \textit{областью определения} функции $f$. Иногда будем обозначать это множество как $D_f$. В таком случае, будем говорить, что функция определена на множестве $G$, если $G \subset D_f$. 
\item
Множество всех $y=f(x)$, $x\in D_f$ называется \textit{множеством значений} функции~$f$ и обозначается $f(D)$ 
$$
f(D_f)=\{y \,\big|\, y=f(x),\, x\in D_f\}.
$$
Отметим, что $f(D_f) \subset Y$, но не обязано совпадать с $Y$.
\item 
Множество всех точек плоскости с координатами $(x, f(x))$, $x\in D_f$, называется \textit{графиком} функции~$f$ 
$$
\Gamma_f = \{(x,y) \,\big|\, y = f(x),\, x\in D_f\}.
$$
\item
Если $M \subset D_f$, то множество всех $y=f(x)$, когда $x\in M$, называется \textit{образом множества} $M$ и обозначается $f(M)$ $$f(M) = \{y\,\big|\, y=f(x),\; x\in M\}.$$
\item
Если $E \subset f(D_f)$, то множество всех $x$, когда $y\in E$, называется \textit{полным прообразом} множества $E$ и обозначается $f^{-1}(E)$ 
$$
f^{-1}(E)=\{x \,\big|\, x\in D_f,\; f(x)\in E\}.
$$
При этом для каждой пары $(x; y=f(x))$, $x\in D_f$ элемент $x$ называется \textit{частным прообразом} точки $y$.
\item
При $S\subset D_f$ функция $f_S\colon S\to\bbR$, при $f_S(x)=f(x)$ называется \textit{сужением (ограничением, следом)} функции $f$ на $S$ и обозначается $f\bigr|_S$.
\end{itemize}
 
Наряду с термином <<функция>> в определенных ситуациях употребляются равнозначные ему термины <<отображение>>, <<преобразование>>, <<морфизм>>, <<соответствие>>.
\begin{notion}
Используя определение функции можно дать еще одно альтернативное определение для понятия <<последовательность>>. Всякое отображение $f\colon \bbN \to Y$  называется \textit{последовательностью} $\{y_n\}$ элементов множества $Y$, где ее $n$-ым элементом будет являться $(n, f(n))$.
\end{notion} 

\begin{defn}
Функция $f$ называется \textit{ограниченной сверху (снизу) на множестве} $X\subset D_f$, если множество $f(X)$ ограничено сверху (снизу), т.е.~существует число~$M$ такое, что
$$
\forall x \in X \quad f(x) \le M \quad (\text{соотв., }f(x)\ge M).
$$
\end{defn}

Функция называется \textit{ограниченной на множестве}~$X$, если она на~$X$ ограничена и сверху, и снизу. Функция~$f$ называется \textit{ограниченной} (без указания множества), если она ограничена на всей области. Попросту говоря, функция ограниченна, если множество ее значений $f(D_f)$ ограничено.

\begin{defn}
Пусть функция $f$ определена на множестве $X\subset D_f$. Тогда 
$$
\sup_{X}\limits f \triangleq \sup f(X)\quad (\inf_{X}\limits f \triangleq \inf f(X))
$$ называется \textit{верхней (нижней) гранью числовой функции~$f$} на множестве~$X$. Также может обозначаться как $\sup_{x\in X}\limits f(x)$ (соотв., $\inf_{x\in X}\limits f(x)$)
\end{defn}

Аналогично, если множество не указано, будем иметь в виду всю область определения функции. Заметим, что согласно предыдущей главе, определение, например, точной верхней грани $M$ числовой функции~$f$ равносильно выполнению следующих двух пунктов 
\begin{enumerate}
\item
$\forall x \in D_f\quad f(x) \le M$,
\item
$\forall M' < M \quad \exists x\in D_f\cquad f(x) > M'$.
\end{enumerate}

\section{Предельная, внутренняя, изолированная точки множества}
Пусть $G \subset \bbR$ "--- множество.
\begin{defn}
Точка $x_0 \in G$ называется \textit{изолированной точкой множества}\rindex{точка множества!изолированная}~$G$, если $$\exists \delta >0\cquad  O_\delta(x_0) \cap G = x_0.$$
\end{defn}

\begin{defn}
Точка $x_0 \in G$ называется \textit{внутренней точкой множества}\rindex{точка множества!внутренняя}~$G$, если 
$$
\exists \delta >0\cquad O_\delta(x_0) \subset G.
$$
\end{defn}

\begin{defn}
Если $c \in \bbR$, то 
\begin{enumerate}[wide, labelwidth=!, labelindent=0pt, nolistsep]
\item
\textit{проколотой $\epsilon$-окрестностью числа} $c$ называется 
$$
\dneio{\epsilon}{c} \triangleq O_\epsilon(c)\setminus\{c\}=(c-\epsilon;c)\cup(c;c+\epsilon);
$$
\item
\textit{проколотой окрестностью точки}~$c$ называется $$\dnei{c} \triangleq O(c)\setminus \{c\};$$
\item[\textbullet]
Если $c$ --- бесконечно удаленная точка, то, $\dneio{\epsilon}{c}\triangleq O (c) $.
\end{enumerate}
\end{defn}

\begin{defn}
Точка $c \in \bboR$ называется \textit{предельной точкой множества}\rindex{точка множества!предельная}~$G$, если $$\forall \epsilon >0 \quad \dneio{\epsilon}{c} \cap G \ne \emptyset.$$
\end{defn}
Попросту говоря, точка $x_0 \in \bbR$ является предельной точкой множества~$X \subset \bbR$, если в любой ее окрестности существует хотя бы одна точка из~$G$, отличная от~$x_0$. По аналогии с этим, если множество~$G$ является неограниченным сверху, то бесконечно удаленная точка~$+\infty$ тоже является предельной точкой множества $G$, так как в этом случае в любой окрестности этой точки существует точка множества $G$. Если же $G$ не ограничена снизу, то точка $-\infty$ также является предельной точкой множества $G$.

\section{Предел функций}
\begin{defn}[Определение предела функции по Коши\rindex{предел!функции по Коши}]\label{df:ch2:predelCaushi}
Пусть задана функция $f$ на множестве $D_f$. И пусть $c$ "--- конечная или бесконечно удаленная точка из $\bboR$, $x_0$ "--- конечная или бесконечно удаленная предельная точка  множества $D_f \subset \bbR$. Тогда элемент~$c$ называется \textit{пределом функции} $f$ при $x \to x_0$ (или в точке $x_0$), если
$$
\forall O(c)\quad \exists O(x_0)\cquad \forall x \in \dnei{x_0}\cap D_f\quad f(x) \in O(c),
$$  
что эквивалентно
$$
\forall \epsilon > 0\quad \exists \delta >0\cquad \forall x \in \dneio{\delta}{x_0} \cap D_f\quad f(x) \in O_\epsilon (c). 
$$
В этом случае будем писать $\lim_{x \to x_0}\limits f(x) = c$, или $f(x)\to c$ при $x \to x_0$.
\end{defn}
Если точка $x_0$ "--- не предельная для множества~$D_f$, то функция~$f$ не может иметь предела по Коши в точке $x_0$. Также заметим, что в определении предела функции $f$ не требуется, чтобы функция была определена в точке~$x_0$.

\begin{defn}
Если $x_0$ "--- предельная точка области определения~$D_f$ функции $f$, то последовательность $\{x_n\}$ называется \textit{последовательностью Гейне} функции $f$ в точке $x_0$ при условии, что
\begin{enumerate}
\item $\forall n \in \bbN\quad x_n\in D_f\setminus \{x_0\}$;
\item $\lim_{n \to \infty}\limits x_n = x_0$.
\end{enumerate}
\end{defn}

\begin{thm}
\label{th:ch2:predeltochka}
Элемент $c$ является предельной точкой $G\subset \bbR$ (конечная или бесконечно удаленная) тогда и только тогда, когда существует последовательность ${x_n}$, такая что $\forall n \in \bbN$ $x_n \in G\setminus\{c\}$ и $\lim_{n \to \infty}\limits x_n = c$. 
\end{thm}

Согласно теореме~\ref{th:ch2:predeltochka} у функции существует последовательность Гейне в любой точке $x_0$, являющейся предельной для $D_f$.

\begin{defn}[Определение предела функции по Гейне\rindex{предел!функции по Гейне}]\label{df:ch2:predelGeine}
Пусть задана функция $f$ на множестве $D_f$. Пусть $c$ "--- конечная или бесконечно удаленная точка из $\bboR$, $x_o$ "--- конечная или бесконечно удаленная предельная точка  множества $D_f \subset \bbR$. Тогда элемент~$c$ называется \textit{пределом функции} $f$ при $x \to x_0$ (или в точке $x_0$), если для любой последовательности Гейне $\{x_n\}$ функции $f$ в точке $x_0$ выполняется условие 
$$\lim_{n \to \infty}\limits f(x_n) = c.$$ 
И в этом случае будем писать $\lim_{x \to x_0}\limits f(x) = c$, или $f(x)\to c$ при $x \to x_0$.
\end{defn}

\begin{thm}
Определение~\ref{df:ch2:predelCaushi} и определение \ref{df:ch2:predelGeine} эквивалентны.
\end{thm}

\begin{defn}
Пусть $f\colon D_f \to \bbR$ и $S \subset D_f$, $x_0$ --- предельная точка множества $S$, $x_0 \in \bboR$. Тогда элемент $c \in \bbR$ называется \textit{пределом функции~$f$ по множеству $S$} при $x \to x_0$, если $\lim_{x \to x_0}\limits g(x) = c$, где $g(x)=\left.f\right|_S = \{(x;f(x)),x \in S\}$.
\end{defn}

\begin{defn}
\textit{Пределом слева функции $f$} в точке $x_0 \in \bbR$ называется предел сужения функции $f$ на множество $D_f \cap (-\infty;x_0)$ (при условии, что  $x_0$ --- предельная точка $D_f \cap (-\infty;x_0)$). Обозначается $f(x_0-0)=\lim_{x \to x_0-0}\limits f(x)$.
Аналогично определяется \textit{предел справа функции} $f$: $f(x_0+0)=\lim_{x \to x_0+0}\limits f(x)$. 
\end{defn}

\begin{thm}[о трех функциях]\label{th:ch1:threefuncs}
Пусть точка $x_0$ является предельной для областей определения функций $f$, $g$ и $h$. Тогда если существуют пределы $\lim_{x \to x_0}\limits f(x)$, $\lim_{x \to x_0}\limits h(x)$, такие что
$$\lim_{x \to x_0}\limits f(x) = \lim_{x \to x_0}\limits h(x) = A \in \bboR,$$
и при этом выполняется
$$
\exists \delta > 0\cquad \forall x \in \dneio{\delta}{x_0}\quad f(x)\le g(x) \le h(x),
$$
то существует и предел функции $g$ при $x \to x_0$, равный $\lim_{x \to x_0}\limits g(x) = A$.
\end{thm}

\section{Непрерывность функции}
Пусть $G \subset \bbR$. Тогда $\forall x_0 \in G$ выполняется одно и только одно из двух утверждений:
\begin{enumerate}
\item $x_0$ "--- изолированная точка, т.е.\ $\exists \delta>0\cquad \dneio{\delta}{x_0} \cap G = \emptyset$;
\item $x_0$ "--- предельная точка, т.е.\ $\forall \delta > 0\quad \dneio{\delta}{x_0}\cap G \neq \emptyset$.
\end{enumerate}

\begin{defn}
Функция $f$ называется \textit{непрерывной в точке} $x_0 \in D_f$, предельной для $D_f$, если предел $f(x)$ при $x \to x_0$ существует и равен $$\lim_{x\to x_0}\limits f(x)=f(x_0).$$ В любой изолированной точке множества $D_f$ функция $f(x)$ считается \textit{непрерывной} по определению. 
\end{defn}

\begin{defn}
Функция $f$ называется \textit{непрерывной на множестве}~$X \subset \bbR$, если функция $f$ определена на множестве $X$ и непрерывна в каждой точке множества $X$. 
\end{defn}
\begin{lemm}
Функция $f$ непрерывна на промежутке $[a;b)$ (отрезке $[a;b]$) 
$$
\Longleftrightarrow
\begin{cases}
\forall x_0 \in (a;b)\quad \exists \lim_{x \to x_0}\limits f(x) = f(x_0);\\
\exists \lim_{x \to a+0}\limits f(x) = f(a)\quad \Bigl(\text{и } \exists \lim_{x \to b-0}\limits f(x) = f(b)\Bigr).
\end{cases}
$$  
\end{lemm}

\section{Теорема Вейерштрасса}

\begin{thm} [Вейерштрасса\rindex{теорема!Вейерштрасса}] \label{th:ch2:Veyershtrass}
Функция $f$, непрерывная на отрезке $[a;b]$, ограничена и достигает на нем своих точных верхней и нижней граней.
\end{thm}
\begin{proof}
Пусть $f(x)$ "--- функция, непрерывная на отрезке $[a;b]$. Пусть при этом 
$$
M = \sup_{x \in [a,b]}\limits f(x) \in \bboR
\quad\text{и}\quad
m = \inf_{x \in [a,b]}\limits f(x) \in \bboR.
$$
Покажем, что $M < +\infty$ и что существует такая точка $x^* \in [a;b]$, что $f(x^*)= M$.

Итак, $M = \sup_{x \in [a;b]}\limits f(x)$, что по определению значит 
$$
\forall\epsilon > 0\quad \exists x\in [a;b]\cquad M-\epsilon < f(x) \le M. 
$$ 

Это эквивалентно тому, что 
$$
\forall n \in \bbN\quad \exists x_n \in [a;b]\cquad M-\frac{1}{n} < f(x_n) \le M.
$$

Поскольку $ a\le x_n \le b$\quad $\forall n \in \bbN$, то последовательность $\{x_n\}$ ограничена. Поэтому по~\hyperref[ch1:th:TBV]{теореме Больцано"--~Вейерштрасса} можно выделить из нее подпоследовательность ${x_{n_k}}$, сходящуюся к некоторому $x^*$ при $k \to \infty$.

Переходя к пределу в неравенстве $a\le x_{n_k}\le b$, получаем, что $x^* \in [a;b]$. Следовательно, в силу непрерывности функции $f$ на отрезке $[a;b]$ она непрерывна в точке $x^*$ этого отрезка, значит, поскольку $\lim_{k \to \infty}\limits x_{n_k} = x^*$, то  
$$
\lim_{k \to \infty} f(x_{n_k}) = f(x^*).
$$ 

С другой стороны $\forall k \in\bbN$ $M-\frac{1}{n_k} < f(x_{n_k}) \le M$. Переходя к пределу при $k \to \infty$, получаем
$$
\lim_{k \to \infty} f(x_{n_k}) = M.
$$

Из последних двух соотношений получаем $f(x^*) = M = \sup_{x \in [a,b]}\limits f(x)$. 

Отсюда следует, во-первых, что  $M<+\infty$, т.е.\ функция $f$ ограничена сверху, и, во-вторых, что функция $f$ достигает своей верхней грани в точке~$x^*$.

Аналогично можно доказать, что непрерывная на отрезке функция ограничена снизу и достигает на нем своей нижней грани.

Теорема доказана.   
\end{proof}